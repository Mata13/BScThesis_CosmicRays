%---------------------------------------------------------------------      
    \section{Conclusiones}
    %------------------------------ SLIDE ---------------------------------------
    \begin{frame}{Conclusiones} % cada entorno frame es una diapositiva
        \justifying % para justificar el texto, siempre al inicio de cada frame
        % Añade espacio para mover el bloque hacia arriba
        \vspace*{-1.5cm} % Ajusta este valor según sea necesario

        \begin{columns}
            \begin{column}{1.0\textwidth} % Columna izquierda para la lista
                \begin{itemize}
                    \item Se \textbf{obtuvo} el espectro de partículas secundarias a nivel de \emph{Sierra Negra} y a nivel del mar, en el \emph{Puerto de Veracruz}.\\
                    \item En \emph{Sierra Negra} se produjeron más partículas, en \textbf{un orden de magnitud mayor} que en el \emph{Puerto de Veracruz}. 
                    \item Los resultados mostraron que el flujo de partículas se va \textbf{incrementando} a medida que la altitud es mayor, lo que coincide con lo \textbf{observado por el MMN}.
                    \item El \textbf{flujo} de protones y neutrones se \textbf{incrementa} a medida que la altura es mayor.
                \end{itemize}
            \end{column}
        \end{columns}         
    \end{frame} 

    %------------------------------ SLIDE ---------------------------------------
    \begin{frame}{} % cada entorno frame es una diapositiva
        \justifying % para justificar el texto, siempre al inicio de cada frame
        % Añade espacio para mover el bloque hacia arriba
        \vspace*{-1.5cm} % Ajusta este valor según sea necesario

        \begin{columns}
            \begin{column}{1.0\textwidth} % Columna izquierda para la lista
                \begin{itemize}
                    \item La principal \textbf{fuente de ruido} para los detectores en \emph{Sierra Negra} son los \textbf{positrones} y \textbf{electrones}, ya que se producen en una cantidad muy parecida a la de los \textbf{muones}.
                    \item Observamos que CORSIKA tiene \textbf{limitaciones} para seguir partículas con energías $\bm{<} \mathbf{300}$ \textbf{MeV}.
                    \item Para estudiar partículas con bajas energías $<300$ MeV se puede hacer uso de \textbf{EXPACS}.
                \end{itemize}
            \end{column}
        \end{columns}         
    \end{frame}